\chapter{Appendix} \label{ch:appendix}

\section{Open-Source Code} \label{sec:appendix:opensource}
Most of the source code used for these methods is available publically through the following repositories. Each project is licensed under the BSD 3-clause license.

\begin{center}
    \begin{tabular}{ | p{25mm} | p{70mm} | p{70mm} |}
    \hline
    \textbf{Repository} & \textbf{Description} & \textbf{URL} \\
    \hline
    \centering \texttt{llops} & GPU-acceleration library with numpy-like syntax and interchangeable backends. Also supports auto-differentiation and latex rendering through operators submodule. & https://github.com/zfphil/llops \\
    \hline

    \centering \texttt{comptic} & Computational microscopy library for python, including common functions for simulating optical systems. Also included scripts for acquisition using micro-manager. & https://github.com/zfphil/comptic \\
    \hline

    \centering \texttt{pydpc} & Python implementation of DPC algorithm, including simulation, processing real data, and aberration recovery. Also included analysis tools for analyzing SNR in DPC systems.& https://github.com/zfphil/pydpc \\
    \hline

    \centering \texttt{pyfpm} & Python implementation of FPM algorithm, including simulation, processing real data, aberration recovery, and source recovery. & https://github.com/zfphil/pyfpm \\
    \hline

    \centering \texttt{pydeblur} & Python implémentation of motion deblur algorithm. Also includes system analysis tools for convolutional forward models. & https://github.com/zfphil/pydeblur\\
    \hline

    \centering \texttt{illuminate} & LED array firmware with standardized API & https://github.com/zfphil/pydeblur\\
    \hline

    % \centering \texttt{illuminate \n controller} & High-level interfaces for controlling LED arrays, including MATLAB, python, micro-manager, and GUI interfaces. & https://github.com/zfphil/illuminatecontroller\\
    % \hline

    \centering \texttt{compcs} & CAD files and Android application for Computational CellScope. & https://github.com/zfphil/compcs\\
    \hline

    \end{tabular}
\end{center}

\section{Quantitative Phase Imaging}\label{sec:appendix:phase}

\subsection{DPC System Parameters}\label{sec:appendix:dpc_sys_param}
\begin{center}
    \begin{tabular}{ | c | c |}
    \hline
    \textbf{Parameter} & \textbf{Value} \\
    \hline
    Objective Mag / NA& $10\times / 0.25NA$\\
    \hline
    Illumination Power & See table~\ref{sec:appendix:dpc_led_power}\\
    \hline
    Camera Readout Noise & 3.7 $e^-$ \\
    \hline
    Camera Quantum Efficiency & $60\%$ \\
    \hline
    Camera Pixel Size & $6.5\mu m$\\
    \hline
    Number of Brightfield LEDs & $72$\\
    \hline
    \end{tabular}
\end{center}

% aptf_green : 4666.4586 Lux
% aptf_blue : 3499.8440 Lux
% aptf_red : 2738.1699 Lux
% vcc_white : 2236.8149 Lux
% thorlabs_blue : 1156.9732 Lux

\subsection{LED Illuminance Values for DPC Analysis}\label{sec:appendix:dpc_led_power}
This section describes measured LED power values for a range of common LEDs. Each of these measurements was made at the sample plane using an optical power-meter (Thorlabs PM100D) at a distance of 50$mm$ from the LED.
\begin{center}
    \begin{tabular}{ | p{35mm} | p{20mm} | p{35mm} | p{70mm} |}
    \hline
    \textbf{\centering LED Mfg. \newline Part Number} & \textbf{LED Type} & \textbf{Illuminance} & \textbf{Comment} \\
    \hline
    \centering VCC\newline VAOL-3LWY4 & Through-hole & 1000.42 Lux & Used in Computational CellScope~\cite{phillips2015multi}\\
    \hline
    \centering Knightbright\newline APTF1616SEEZ & SMD & (Red) 2738 Lux \newline (Green) 4666 Lux \newline (Blue) 3500 Lux & Used for Quasi-Dome~\cite{phillips2017quasi}\\
    \hline
    \centering VCC\newline VAOL-3LWY4 & Through-hole & 1000.42 Lux & Used for High-Throughput Fluorescence Imaging (Chapter~\ref{ch:highthroughput})\\
    \hline
    \end{tabular}
\end{center}


\section{High-Throughput System Analysis}
\label{sec:appendix:hightroughput}
\subsection{Derivation of SNR Expression}

\label{sec:appendix:snr_derivation}

In this section we derive the expression for the SNR of a recovered image. Considering the additive noise acquisition model, $\y = \A\x + \noise$,
the recovered image $\widehat \x$ is given by:
\[\widehat \x =  \A^\dagger  \y = \x +  \A^\dagger  \noise\:.\]
In what follows, we assume only that $\noise$ is zero mean with covariance $\sigma_\noise^2 \I$,

Defining the mean of the recovered object $\mu = \mathbb{E}[\widehat \x]$, as well as the covariance $\mat\Sigma = \mathbb{E}[(\widehat \x-\x) (\widehat \x-\x)^\top]$, we calculate the imaging SNR using the root mean squared error (RMSE):
\[SNR = \frac{\frac{1}{m}\sum_{i=1}^m\mu_i}{\sqrt{\frac{1}{m}\mathrm{Tr}(\mat\Sigma)}}\:.\]
Assuming zero-mean noise, the numerator is simply the average object signal $\bar s$.
Expanding the covariance term in the denominator,
\begin{align*}
    \Sigma &= \mathbb{E}[ \A^\dagger  \noise( \A^\dagger  \noise)^\top]
    %\\
    %&= \mathbb{E}[ A^\dagger  \eta\eta ^\top ( A^\dagger)^\top]
    = \sigma_{ \noise}^2  \A^\dagger ( \A^\dagger)^\top\:,
\end{align*}
where we apply the assumption that the covariance of $\noise$ is $\sigma_\noise^2 \I$. Then,
\[\mathrm{Tr}( \A^\dagger ( \A^\dagger)^\top ) = \sum_{i=1}^m \frac{1}{\sigma_i( \A)^2} = \frac{1}{\sigma_1(\A)^2} \sum_{i=1}^m \frac{\sigma_1(\A)^2}{\sigma_i(\A)^2} \:.\]
Thus we have that
\[SNR = \frac{\bar s}{\frac{1}{\sigma_1(\A)} \sqrt{\frac{1}{m}\sum_{i=1}^m \frac{\sigma_1(\A)}{\sigma_i(\A)}} \cdot\sigma_\eta} := \frac{\sigma_1(\A) \bar s }{ f\sigma_\eta }\:,\]
where $f$ is the general definition of the deconvolution noise factor. This expression is consistent with the definition in~\eqref{eq:DNF} for convolutional operators, where we note that the singular values are given by the power spectrum of the kernel $\h$. Further, we note that in this case $\sigma_1(\A)=\gamma$ since that is the DC component of a non-negative signal.

\subsection{Multi-frame Decomposition}\label{sec:appendix:multiframe_app}

We consider the case of a multiframe operator with the same blur kernel $\h$ used in every frame. In this case, the forward operator has the form
\begin{equation*}
\A =  \begin{bmatrix}\W_1 \\ \vdots \\ \W_n \end{bmatrix}\B := \W\B\:.
\end{equation*}
 Following the derivation of SNR from the previous section, we compute $\mathrm{Tr}(\A^\dagger (\A^\dagger)^\top)$. First,
\[\A^\dagger = (\B^\top {\W}^\top{\W} \B)^{-1} \B^\top {\W}^\top =
\B^{-1} ({\W}^\top{\W} )^{-1} {\W}^\top\:,\]
assuming that $\B$ and $\W^\top \W$ are invertible.
Then we have that
\begin{align*}
    \mathrm{Tr}(\A^\dagger &(\A^\dagger)^\top) = \mathrm{Tr}(\B^{-1} ({\W}^\top{\W} )^{-1} {\W}^\top {\W} ({\W}^\top{\W} )^{-\top}
    \B^{-\top}) \\
    &= \mathrm{Tr}(\B^{-1} ({\W}^\top{\W} )^{-1}
    \B^{-1}) = \mathrm{Tr}(({\W}^\top{\W} )^{-1} \B^{-2} )\:.
\end{align*}
We now consider the form of ${\W}^\top{\W} = \sum_{j=1}^n \W_j^\top \W_j$. Each $\W_j^\top \W_j$ is a square diagonal matrix with either a $0$ or $1$ for each diagonal entry, depending on whether the corresponding pixel is included in the window. Thus the sum ${\W}^\top {\W}$ is a diagonal matrix with the $i$th diagonal value given by the number of times pixel $i$ is included in the windows $\{\W_1,...,\W_n\}$, a quantity we denote as $c_i = \sum_{j=1}^n \W_j \e_i$ where $\{\e_i\}$ are the standard basis vectors.
% \begin{align*}
%     \mathrm{Tr}(A^\dagger &(A^\dagger)^\top) = \mathrm{Tr}(B^{-2} \mathrm{diag}(c_1^{-1},...,c_m^{-1}))\:.
% \end{align*}

Before we proceed further, note that for any matrices $\M$ and $\D$ with non-negative entries and $\D$ diagonal,
\[\mathrm{Tr}(\D\M) = \sum_{i} D_{ii} M_{ii}  \leq \max_{i} D_{ii} \cdot \mathrm{Tr}(\M)\:.\]
% Then we have that
% \begin{align*}
% \kappa(A) &= \kappa({W} B) \\&\leq \kappa({W}) \cdot \kappa(B) = \frac{\max_{j}c_j}{\min_{j}c_j} \cdot \frac{\max_i |\tilde h|_i}{\min_i |\tilde h|_i}\:.
% \end{align*}
% Since the condition number is an upper bound on the DNF, this provides a lower bound on SNR for this multiframe case.
We can therefore conclude that
\begin{align*}
    \mathrm{Tr}(\A^\dagger (\A^\dagger)^\top)&\leq \max_{i}\frac{1}{c_i}\cdot \mathrm{Tr}(\B^{-2})
    =
    \frac{1}{\min_{i} c_i}   \cdot \sum_{i=1}^m \frac{1}{|\tilde \h|_i^2}\:.
\end{align*}
Thus we see that the expression for the covariance is decreased by a factor of at least the square root of minimum coverage. This corresponds to the lower bound on the SNR:
\[SNR \geq \sqrt{{\min_{i} c_i}}\cdot \frac{ \gamma\bar s}{ f \sigma_\eta  }\:.\]
Where $f$ is defined as in~\eqref{eq:DNF}.



\subsection{Blur Kernel Optimization} \label{sec:appendix:optimization_app}

In this section we discuss the reformulation of
the optimization problem in~\eqref{eq:illum_opt_single} as a smooth objective with convex constraints.
Recall that the optimization problem has the form
\begin{align*}
\min_{\h}&~~ \sqrt{\frac{1}{m} \sum_{i=0}^m \frac{\max_i{|\tilde{\h}|_i^2}}{|\tilde{\h}|_i^2}} \\
  s.t. &~~0 \leq h_i \leq 1 \; \forall \; i, \quad
  \sum_{i} h_i = \gamma \:.
\end{align*}
First, note that by definition $\tilde \h=\F\h$ where $\F$ represents the discrete Fourier transform (DFT) matrix.
Then, we know that $\max_i{|\tilde{\h}|_i^2}$ is the DC component of the signal, which is equal to $\gamma$ and therefore fixed for any feasible $\h$.
Therefore, the blur kernel which maximizes~\eqref{eq:illum_opt_single} is the same as the one that maximizes
\begin{align*}
\label{eq:opt_smooth}
\min_{\h}&~~ \sum_{i=0}^m \frac{1}{(\F_i^\top\h)^2} ~~
  s.t. ~~0 \leq h_i \leq 1, ~
  \sum_{i} h_i = \gamma \:,
\end{align*}
where $\F_i$ represents columns of the DFT matrix.

It is possible to use projected gradient methods because the objective function is smooth nearly everywhere and the constraints are convex. At each iteration, there is a gradient step followed by a projection step.
The gradient step is defined as
\begin{align*}
    \tilde \h^{k+1} = \h^k + \alpha^k \sum_{i=0}^m   \frac{2}{(\F_i^\top \h^k)^{3}} \cdot \F_i\:,
\end{align*}
for potentially changing step size $\alpha^k$.
%\nabla \sum_{i=0}^m \frac{1}{(F_i^\top h)^2} = \sum_{i=0}^m   -2(F_i^\top h)^{-3} \cdot \nabla (F_i^\top h)=
The projection step is defined as
\begin{align*}
    \h^{k+1} = \Pi_{\mathcal{S}}(\tilde \h_{k+1})\:,
\end{align*}
where $\mathcal{S}$ is the intersection of the box constraint $\{0\leq h_i\leq 1\}$ and the simplex constraint $\{\sum_i h_i = \gamma\}$. Efficent methods for this projection exist~\cite{gupta2010l1}.

\subsection{Fundamental DNF Limits}\label{sec:appendix:dnf_limit}

There are fundamental limits on how SNR can be improved by coded illumination.
We examine a fundamental lower bound on the DNF to demonstrate this.

Recall that
\[f^2 = \max_i |\tilde \h|_i^2\cdot \frac{1}{m} \sum_{i=1}^m \frac{1}{|\tilde \h|_i^2}\:.\]
Then, note that $\frac{1}{m} \sum_{i=1}^m \frac{1}{|\tilde \h|_i^2}$ is the reciprocal of the {harmonic mean} of $\{|\tilde \h|_1^2,...,|\tilde \h|_m^2\}$. Since the harmonic mean is always less than the {arithmetic mean}, we have that
$$ \frac{1}{m} \sum_{i=1}^m \frac{1}{|\tilde \h|_i^2}\geq \frac{1}{\frac{1}{m}\sum_{i=1}^m |\tilde \h|_i^2}\:. $$

Next, we apply Parseval's and have $\frac{1}{m}\sum_{i=1}^m |\tilde \h|_i^2 = \sum_{i=1}^m h_i^2$.
Additionally, $\max_i |\tilde \h|_i$ is the DC component of the signal, which is specified by the constraint $\sum_{i=1}^m h_i = \gamma$.
As a result, $$ f^2 \geq \frac{\gamma^2}{\sum_{i=1}^m h_i^2}\:.$$
Finally, we see that
$$\max_{\h\in[0,1]^m} \sum_{i=1}^m h_i^2~:~ \sum_{i=1}^m h_i=\gamma$$
is achieved for binary $h$ and has the maximum value $\gamma$. Therefore,
\[f^2 \geq \frac{\gamma^2}{\sum_{i=1}^m h_i^2} \geq \frac{\gamma^2}{\gamma} = \gamma\:.\]
% \begin{lem}
% We have the following relationship between average squared singular value and illumination values, $\frac{1}{m}\sum_{i=1}^m |\tilde h|_i^2 = \sum_{j=1}^d v_j^2$.
% \end{lem}
% \begin{proof}
% First, note that $\tilde h = Fh = F\sum_{j=1}^d v_j \delta_j$. Then the average squared singular values is given by the inner product
% $$\frac{1}{m} \sum_{i=1}^m |\tilde h|_i^2 = \frac{1}{m} (Fh)^H Fh =  \frac{1}{m} \sum_{j=1}^n \sum_{\ell=1}^n v_jv_\ell \delta_j^H F^H F\delta_\ell =  \sum_{j=1}^n  v_j^2 $$
% where the final simplification comes from noting that $F^H F = mI$ (since we use un-normalized DFT matrices, TODO) and $\delta_j^H\delta_\ell = \mathbf{1}\{j=\ell\}$.
% \end{proof}
That is, the DNF grows at a rate of at least $\sqrt{\gamma}$. As a result,  the best achievable SNR (using~\eqref{eq:snr_coded}) is
$$ SNR  \leq \frac{\sqrt{\gamma}\bar{s}_0}{\sqrt{\gamma\bar{s}_0 + \sigma^2_{r}}}
=\sqrt{\bar{s}_0}\sqrt{\frac{\gamma \bar{s}_0 }{\gamma\bar{s}_0 + \sigma^2_{r}}}\:,
$$
This upper bound on SNR increases with ${\gamma}$.
In Methods Section~\ref{sec:highthroughput:illum_opt}, we discuss an exact closed form for $f(\gamma)$ that yields an expression for optimal multiplexing.

However, if $\sigma_r$ is much smaller than the total captured signal, i.e. $\sigma_r \ll \gamma \bar{s}_0 $, the SNR will not increase with $\gamma$, and in fact its maximum value,
$$SNR \leq \sqrt{\bar{s}_0}$$
is achieved by strobed illumination (i.e. $\gamma=1$). In other words, when signal is large compared with readout noise, strobed will be optimal, regardless of the illumination optimization method.


\subsection{Derivation of Illumination Throughput}\label{sec:appendix:app_throughput}

\subsubsection{Stop-and-Stare}
In the stop-and-stare acquisition strategy, the sample is illuminated for the full dwell time ($t_{sns}$), which is set by motion stage parameters such as maximum velocity, acceleration, and the necessary stage settle time ($v_{stage}$, $a_{stage}$, and $t_{settle}$ respectively), as well as camera readout ($t_{readout}$). These parameters are related to frame rate $r_{frame}$ by the following relationship:
\begin{equation*}
t_{sns} = \frac{1}{r_{frame}} - \max (t_{readout}, 2t_{accel} + t_{move})
\end{equation*}

Note that this equation assumes perfect hardware synchronization and instantaneous acceleration ($\frac{\partial a}{\partial t} = \infty$). The variables $t_{accel}$ and $t_{move}$ are defined as:

\begin{equation*}
t_{accel} = \frac{v_{stage}}{a_{stage}}
\end{equation*}

\begin{equation*}
t_{move} = \frac{d_{frame} - 0.5 * a_{stage} * t_{accel}^2}{v_{stage}}
\end{equation*}

Here the expression $d_{frame} = FOV * (1-O)$ is the distance between frames, which is determined by the field-of-view of a single frame ($FOV$) and inter-frame overlap fraction $O$.

Combining terms, we arrive at an expression for $t_{sns}$:
\begin{equation*}
t_{sns} = \frac{1}{r_{frame}} - \max (t_{readout}, 2\frac{v_{stage}}{a_{stage}} - \frac{d_{frame} - 0.5 * a_{stage} * t_{accel}^2}{v_{stage}})
\end{equation*}

When camera readout time $t_{readout}$ is short, $t_{sns}$ can be simplified to:

\begin{equation*}
t_{sns} = \frac{1}{r_{frame}} - 2\frac{v_{stage}}{a_{stage}} - \frac{d_{frame} - 0.5 * \frac{v_{stage}^2}{a_{stage}}}{v_{stage}}
\end{equation*}

\subsubsection{Strobed Illumination}

The maximum pulse duration for strobed illumination is related to the time required to move a distance of one effective pixel size $\frac{\Delta}{M}$ at a velocity $v_{stage}$:
\begin{equation*}
t_{strobe} = \frac{\frac{\Delta}{M}}{v_{stage}}
\end{equation*}

The stage velocity $v_{stage}$ may be bounded by the motion stage hardware ($v_{max}$) or by the field of view of the microscope ($FOV$):

\begin{equation*}
v_{stage} = \min (v_{max}, r_{frame}FOV)
\end{equation*}

\subsubsection{Coded Illumination}
The calculation of $t_{coded}$ for coded illumination is synonymous to the strobed illumination case, weighted by the multiplexing coefficient used to generate the illumination sequence ($\gamma$), and using the $v_{stage}$ calculation from the strobed subsection:
\begin{equation*}
t_{coded} = \frac{\frac{\gamma \Delta}{M}}{v_{stage}}
\end{equation*}

\subsection{System Parameters}\label{sec:appendix:sys_param}
% \scalebox{0.7}{
    \begin{center}
    \begin{tabular}{ | c | c |}
    \hline
    \textbf{Parameter} & \textbf{Value} \\
    \hline
    Maximum Motion Stage Velocity & $40\frac{m}{s}$ \\
    \hline
    Motion Stage Acceleration & $400 \frac{m}{s^2}$\\
    \hline
    Motion Stage Settle Time & $0.1s$\\
    \hline
    Objective Mag / NA& $10\times / 0.25NA$\\
    \hline
    Frame Overlap & $20\%$ \\
    \hline
    Illumination Power & $600$ \\
    \hline
    Camera Readout Time & 26ms\\
    \hline
    Camera Readout Noise & 3.7 $e^-$ \\
    \hline
    Camera Quantum Efficiency & $60\%$ \\
    \hline
    Camera Pixel Size & $6.5\mu m$\\
    \hline
    Fluorophore Quantum Yield& $79\%$ lux\\
    \hline
    Illumination repetition Rate & $250 kHz$\\
     \hline
    \end{tabular}
    \end{center}
% }
