\begin{center}
    \textbf{Abstract}
    
    \vspace{1.2cm}
     Quantitative Optical Microscopy Using Coded Illumination
 
    \vspace{0.4cm}
    
    By
 
    \vspace{0.4cm}
    Zachary F. Phillips
    
     \vspace{0.4cm}
     Doctor of Philosophy in Applied Science and Technology
     
     \vspace{0.4cm}
     University of California, Berkeley
     
     \vspace{0.4cm}
     Ted Van Duzer Associate Professor Laura Waller, Chair
 
    \vspace{0.8cm}

    
\end{center}

Quantitative optical microscopy continues to be a powerful tool in biology and throughout the sciences. Computational microscopy blends large-scale computation with conventional image formation principles to enable a large number of imaging modalities not possible with conventional techniques, such as quantitative phase imaging, motion deblurring, and super-resolution techniques. In this work, we present several novel examples of computational microscopy using a programmable illumination source such as a LED array to introduce known images which have been distorted by a know mathematical transformation. We first introduce quantitative phase imaging using differential phase contrast, which uses multiple measurements to recover the linearized complex field of a thin, transparent sample, and demonstrate a novel single-shot variant using color-multiplexing. Second, we explore coded illumination for high-throughput imaging, and demonstrate a temporal-coding technique which enables significantly higher SNR for high-speed slide scanning and neuropathology applications. We examine the various hardware elements which limit acquisition speeds, and provide a framework for defining when our method is advantageous over existing techniques. For fluorescent imaging, we demonstrate a 10$\times$ improvement in reconstruction SNR compared to conventional high-speed imaging techniques. Next, we explore the design and fabrication of LED illumination devices, including a quasi-dome LED illuminator which enables high-angle illumination for a variety of applications. Finally, we demonstrate two examples of self-calibration techniques for computational imaging systems employing coded illumination. These include aberration recovery using differential phase contrast and source calibration for the quasi-dome LED array, using both image-based calibration and an online method based on Fourier ptychography.